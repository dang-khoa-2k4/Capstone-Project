% 
\section{Motivation}
Trajectory optimization offers mature tools for motion planning in high-dimensional spaces
under dynamic constraints. However, when facing complex configuration spaces, cluttered
with obstacles, roboticists typically fall back to sampling-based planners that struggle in very
high dimensions and with continuous differential constraints. Indeed, obstacles are the source
of many textbook examples of problematic nonconvexities in the trajectory-optimization prob-
lem. Here we show that convex optimization can, in fact, be used to reliably plan trajectories
around obstacles. Specifically, we consider planning problems with collision-avoidance con-
straints, as well as cost penalties and hard constraints on the shape, the duration, and the
velocity of the trajectory. Combining the properties of Bézier curves with a recently-proposed
framework for finding shortest paths in Graphs of Convex Sets (GCS), we formulate the
planning problem as a compact mixed-integer optimization. In stark contrast with existing
mixed-integer planners, the convex relaxation of our programs is very tight, and a cheap round-
ing of its solution is typically sufficient to design globally-optimal trajectories. This reduces
the mixed-integer program back to a simple convex optimization, and automatically provides
optimality bounds for the planned trajectories. We name the proposed planner GCS, after
its underlying optimization framework. We demonstrate GCS in simulation on a variety of
robotic platforms, including a quadrotor flying through buildings and a dual-arm manipulator
(with fourteen degrees of freedom) moving in a confined space. Using numerical experiments
on a seven-degree-of-freedom manipulator, we show that GCS can outperform widely-used
sampling-based planners by finding higher-quality trajectories in less time.